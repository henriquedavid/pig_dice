This programming project implements a simple clone of the popular game {\bfseries Pass the Pig}\copyright{} and {\bfseries Pig\+Mania}\copyright{}.

The main objective with this assignemnt is to introduce separate compiling, progressive programming (small versions towards the final full-\/fledged program), as well as to provide an oportunity to develop a project using the tools already taught such as git, gdb, and doxygen.

Another objective is to demonstrate the importante of modular programming. For that, I\textquotesingle{}m planning a Pig\textquotesingle{}s tournment where each programming team\textquotesingle{}s IA plays against other teams. For that to happen, the programming teams should write code that strictly complies with the project specifications.

\section*{The Gameplay}

Pig is a folk jeopardy dice game described by John Scarne in 1945, in which two players compete to reach {\bfseries 100 points} first.

The game is organized in turns. Each turn, the same player keeps rolling the dice, while the face points are accmulated into a {\bfseries turn total}, until one of these two conditions happen\+:


\begin{DoxyEnumerate}
\item The current player decides to {\itshape hold}, in which case s/he scores the {\bfseries turn total}; or,
\item The face \&\#9856;, (the pig) is rolled, in which case the current player scores nothing.
\end{DoxyEnumerate}

In both cases, the {\bfseries turn total} is set to zero and it becomes the opponent\textquotesingle{}s turn.

In summary, any time during a player\textquotesingle{}s turn, s/he must take one of these two {\itshape actions}\+:


\begin{DoxyItemize}
\item {\bfseries Roll} the dice, which may produce\+:
\begin{DoxyItemize}
\item Face \&\#9856;\+: the player scores nothing and it becomes the opponent\textquotesingle{}s turn;
\item Faces \&\#9857;, \&\#9858;, \&\#9859;, \&\#9860;, or \&\#9861;\+: the face point is added to the {\bfseries turn total} and the player\textquotesingle{}s turn continues.
\end{DoxyItemize}
\item {\bfseries Hold}
\begin{DoxyItemize}
\item The {\itshape turn total} is added to the player\textquotesingle{}s overall score and it becomes the opponent\textquotesingle{}s turn.
\end{DoxyItemize}
\end{DoxyItemize}

The player who scores {\bfseries 100} or more points at the end of a turn is the winner.

\section*{Authorship}

Program developed by Henrique David ($<$ $\ast$henriquemed101.com$\ast$ $>$), 2017.\+1

\copyright{} I\+M\+D/\+U\+F\+RN 2018. 